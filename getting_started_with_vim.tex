\documentclass{article}
\usepackage[margin=1in]{geometry}
\usepackage[colorlinks=true]{hyperref}
\usepackage{longtable}
\usepackage{booktabs}

\begin{document}
\title{Getting started with vim}
\author{Md Arif Shaikh}
\maketitle
\tableofcontents

\section{Installing Plugins}
\label{sec:installing-plugins}

\section{Python in vim}
\label{sec:python-in-vim}
One of the most important task of my job is to write python code. That is why,
any editor that I choose must be capable of being a good python editor. Bellow
are the things that an editor should be able do provide:

\begin{enumerate}
    \item {\bfseries Autocompletion:} For autocompletion of python code one need
        the \verb!vim-jedi! plugin. To bring up the auto completion one has to
        use \verb!C-space!. Also the autocompletion automatically works after
        dot.\\ 

        \underline{\bfseries Launch mvim from the terminal:}\\
        Thing to keep in mind is that the jedi-vim uses the
        python installed in the system if the vim application is launched from
        say the spotlight in Mac. In that case it will not use the conda
        environment. Therefor it is best to launch it from the command line where you
        are already in a conda env. In that way it will have access and will do
        autocompletion for all the modules and packages installed in the conda env.\\

    \item {\bfseries flake8:} flake8 highlights any deviation from pep8 style
        guideline as well as error in the python code. This could be done in vim
        in the following steps.
    \begin{itemize}
        \item First install flake8 using pip: \verb!pip install flake8!.
        \item Install flake8 plugin in .vimrc: Plugin \verb!`nvie/vim-flake8'!
        \item If vim cannot find flake8 then we need to provide the flake8
            plugin the absolute path of the flake8 installation. For example in
            the path could be obtained in terminal using \verb!which flake8!.
            Now in the \verb!.vim! folder go the folder for \verb!vim-flake8! 
            plugin and then to \verb!autoload! directory where you will find the 
            \verb!flake8.vim! file. In that file we need to change the line
            which calls the flake8 package in the following way:
            \begin{verbatim}
            call s:DeclareOption('flake8_cmd', '','"/absolute/path/to/flake8"')
            \end{verbatim}  
        \item Now that flake8 is currently working, the last thing we want is to 
            run the flake8 command each time we save the python file. This is done
            easily by putting the following line in the \verb!.vimrc!:\\
            \verb!autocmd BufWritePost *.py call Flake8()!.
        \item Often it is nice to have a marker in the file itself and also at
            the begining of the line showing the error or the warning. This
            could be accomplished using the following settings in the \verb!.vimrc!
            \begin{verbatim}
                let g:flake8_show_in_file = 1
                let g:flake8_show_in_gutter = 1
            \end{verbatim}
    \end{itemize}
\item {\bfseries Correct indentation:} Sometimes when inside a bracket a python
    line becomes too long we have make a line break but this might mess the 
    indentation. To rectify the indentation put the cursor on one of the
    brackets and in normal mode and use \verb!=%!.

\item {\bfseries Commentting and uncommenting:} It's always handy to know how to
    comment out multiple line of code. In vim this can be done in the following
    way.
    \begin{enumerate}
        \item start by selecting a block using \verb!Control+v! and using
            \verb!j! or \verb!k! to go down or up.
        \item After the selecting the lines use \verb!Shift+I! to insert the
            desired commenting character. For example in python this would be
            \verb!# !. 
        \item Enter \verb!ESC! and it will do the job.
        \item For uncommenting, select the lines using the first step and use
            \verb!x! to uncommnet.
    \end{enumerate}


\end{enumerate}

\section{Latex in vim}
\label{sec:latex-in-vim}
Apart from programming, an integral part of my academic life is to write
documents for publications of my research and for that I have to write a lot of
latex files. For the same reason, I have to figure out how to vim efficiently to
write \verb!tex! files and compile. Luckily there exist a good Plugin named
\verb!vimtex! which fulfill almost all of my requirements for writing latex files.

Bellow are some commands useful when writing latex files:

\begin{longtable}{ll}
\toprule
    Start compiling on save & \verb!\ll! or :VimtexCompile\\ \midrule
    Show errors & \verb!\le! \\ \midrule
    after writing \verb!\begin{env}! complete it & \verb!]]!\\ \midrule
    set spell checking & \verb!:set spell!\\ \midrule
    correct spell on point & \verb!z=!\\ \midrule
\end{longtable}

\noindent {\bfseries How to use \verb!\ll! in \verb!Normal! mode to view the line on
cursor in the pdf file opened in Skim?:}
This could be accomplished just by adding the following lines in the
\verb!.vimrc! file.

\begin{verbatim}
let g:vimtex_view_general_viewer = '/Applications/Skim.app/Contents/SharedSupport/displayline'
let g:vimtex_view_general_options = '-r @line @pdf @tex'
let g:vimtex_view_general_options_latexmk = '-r 1'   
\end{verbatim}


\section{Snippets in vim}
\label{sec:snippets-in-vim}
To make one's workflow one should use Snippets. In vim this nicely accomplished
using \verb!UltiSnips!. To create and use snippets one should follow the steps
described below.

\begin{enumerate}
    \item First install the plugin using in \verb!.vimrc!. 
        \begin{verbatim}
            Plugin 'SirVer/ultisnips'
        \end{verbatim}
    \item Next make a directory named \verb!UltiSnips! inside \verb!.vim! or
        symlink it there. 
    \item Now let's say we want to create snippets for \verb!.tex! files. We
        create a file named \verb!tex.snippets!. 
    \item We are now ready to create snippets for latex insdie that file.
        template of snippets is the following
        \begin{verbatim}
            snippet eq "equation" b
            \begin{equation}
	        \label{eq:$1}
	        $0
            \end{equation}
            endsnippet
        \end{verbatim}
\end{enumerate}

\section{Diectories in vim}%
\label{sec:diectories_in_vim}
Simply use \verb!:e! to explore file systems. Once in side the nerd tree view,
it's easy to see all the commands using the help me by pressing \verb!?!. Here
are few of the Shortcuts one needs

\begin{tabular}{ll}
    \toprule
    Key & Action \\ \midrule
    \verb!:u! & up a directory\\ \midrule
    \verb!:C-j! & move to next child \\ \midrule
    \verb!:C-k! & move to previous child\\ \midrule
    \verb!m! & open the menu for adding/deleting sud directories.\\ \bottomrule
\end{tabular}


\section{Shortcuts}
\label{sec:shortcuts}

\begin{longtable}{ll}
\toprule
{\bfseries Buffers} & \\ \midrule
Open a new buffer & :new \\midrule
Open a terminal inside vim & :term \\ \midrule
Switch between windows & C-w C-w \\ \midrule
close/unload/delete the current buffer & :bd \\ \midrule
{\bfseries Navigation} & \\ \midrule
Beginning of the buffer & 1G \\ \midrule
End of buffer & G \\ \midrule
Go to nth line & nG\\ \midrule
Beginning of line & 0\\ \midrule
End of line & \$\\ \midrule
Open new line after cursor & o \\ \midrule
Open new line before cursor & O \\ \midrule
Delete the current line & dd \\  \midrule
Move down by one screen & \verb!C-f!\\ \midrule
Move up by one screen & \verb!C-b!\\ \midrule
In the given screen move to top & \verb!H!\\ \midrule
In the given screen move to midlle & \verb!M!\\ \midrule
In the given screen move to the last line & \verb!L!\\ \midrule
{\bfseries Copy \& paste} & \\ \midrule
Copy the current line & \verb!yy!\\ \midrule
Paste the copied line below the cursor & \verb!p!\\ \midrule
Paste the copied line above the cursor & \verb!P!\\ \midrule
{\bfseries Visual marking} & \\ \midrule
start marking & v \\ \midrule
select the region & using arrows\\ \midrule
cut the marked region & d \\ \midrule
copy the marked region & y \\ \midrule
paste the copied/cut text & p \\ \midrule
                          & \\ \bottomrule
\end{longtable}
\end{document}
